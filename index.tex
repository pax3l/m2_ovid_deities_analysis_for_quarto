% Options for packages loaded elsewhere
\PassOptionsToPackage{unicode}{hyperref}
\PassOptionsToPackage{hyphens}{url}
\PassOptionsToPackage{dvipsnames,svgnames,x11names}{xcolor}
%
\documentclass[
]{agujournal2019}

\usepackage{amsmath,amssymb}
\usepackage{iftex}
\ifPDFTeX
  \usepackage[T1]{fontenc}
  \usepackage[utf8]{inputenc}
  \usepackage{textcomp} % provide euro and other symbols
\else % if luatex or xetex
  \usepackage{unicode-math}
  \defaultfontfeatures{Scale=MatchLowercase}
  \defaultfontfeatures[\rmfamily]{Ligatures=TeX,Scale=1}
\fi
\usepackage{lmodern}
\ifPDFTeX\else  
    % xetex/luatex font selection
\fi
% Use upquote if available, for straight quotes in verbatim environments
\IfFileExists{upquote.sty}{\usepackage{upquote}}{}
\IfFileExists{microtype.sty}{% use microtype if available
  \usepackage[]{microtype}
  \UseMicrotypeSet[protrusion]{basicmath} % disable protrusion for tt fonts
}{}
\makeatletter
\@ifundefined{KOMAClassName}{% if non-KOMA class
  \IfFileExists{parskip.sty}{%
    \usepackage{parskip}
  }{% else
    \setlength{\parindent}{0pt}
    \setlength{\parskip}{6pt plus 2pt minus 1pt}}
}{% if KOMA class
  \KOMAoptions{parskip=half}}
\makeatother
\usepackage{xcolor}
\setlength{\emergencystretch}{3em} % prevent overfull lines
\setcounter{secnumdepth}{5}
% Make \paragraph and \subparagraph free-standing
\makeatletter
\ifx\paragraph\undefined\else
  \let\oldparagraph\paragraph
  \renewcommand{\paragraph}{
    \@ifstar
      \xxxParagraphStar
      \xxxParagraphNoStar
  }
  \newcommand{\xxxParagraphStar}[1]{\oldparagraph*{#1}\mbox{}}
  \newcommand{\xxxParagraphNoStar}[1]{\oldparagraph{#1}\mbox{}}
\fi
\ifx\subparagraph\undefined\else
  \let\oldsubparagraph\subparagraph
  \renewcommand{\subparagraph}{
    \@ifstar
      \xxxSubParagraphStar
      \xxxSubParagraphNoStar
  }
  \newcommand{\xxxSubParagraphStar}[1]{\oldsubparagraph*{#1}\mbox{}}
  \newcommand{\xxxSubParagraphNoStar}[1]{\oldsubparagraph{#1}\mbox{}}
\fi
\makeatother


\providecommand{\tightlist}{%
  \setlength{\itemsep}{0pt}\setlength{\parskip}{0pt}}\usepackage{longtable,booktabs,array}
\usepackage{calc} % for calculating minipage widths
% Correct order of tables after \paragraph or \subparagraph
\usepackage{etoolbox}
\makeatletter
\patchcmd\longtable{\par}{\if@noskipsec\mbox{}\fi\par}{}{}
\makeatother
% Allow footnotes in longtable head/foot
\IfFileExists{footnotehyper.sty}{\usepackage{footnotehyper}}{\usepackage{footnote}}
\makesavenoteenv{longtable}
\usepackage{graphicx}
\makeatletter
\def\maxwidth{\ifdim\Gin@nat@width>\linewidth\linewidth\else\Gin@nat@width\fi}
\def\maxheight{\ifdim\Gin@nat@height>\textheight\textheight\else\Gin@nat@height\fi}
\makeatother
% Scale images if necessary, so that they will not overflow the page
% margins by default, and it is still possible to overwrite the defaults
% using explicit options in \includegraphics[width, height, ...]{}
\setkeys{Gin}{width=\maxwidth,height=\maxheight,keepaspectratio}
% Set default figure placement to htbp
\makeatletter
\def\fps@figure{htbp}
\makeatother
% definitions for citeproc citations
\NewDocumentCommand\citeproctext{}{}
\NewDocumentCommand\citeproc{mm}{%
  \begingroup\def\citeproctext{#2}\cite{#1}\endgroup}
\makeatletter
 % allow citations to break across lines
 \let\@cite@ofmt\@firstofone
 % avoid brackets around text for \cite:
 \def\@biblabel#1{}
 \def\@cite#1#2{{#1\if@tempswa , #2\fi}}
\makeatother
\newlength{\cslhangindent}
\setlength{\cslhangindent}{1.5em}
\newlength{\csllabelwidth}
\setlength{\csllabelwidth}{3em}
\newenvironment{CSLReferences}[2] % #1 hanging-indent, #2 entry-spacing
 {\begin{list}{}{%
  \setlength{\itemindent}{0pt}
  \setlength{\leftmargin}{0pt}
  \setlength{\parsep}{0pt}
  % turn on hanging indent if param 1 is 1
  \ifodd #1
   \setlength{\leftmargin}{\cslhangindent}
   \setlength{\itemindent}{-1\cslhangindent}
  \fi
  % set entry spacing
  \setlength{\itemsep}{#2\baselineskip}}}
 {\end{list}}
\usepackage{calc}
\newcommand{\CSLBlock}[1]{\hfill\break\parbox[t]{\linewidth}{\strut\ignorespaces#1\strut}}
\newcommand{\CSLLeftMargin}[1]{\parbox[t]{\csllabelwidth}{\strut#1\strut}}
\newcommand{\CSLRightInline}[1]{\parbox[t]{\linewidth - \csllabelwidth}{\strut#1\strut}}
\newcommand{\CSLIndent}[1]{\hspace{\cslhangindent}#1}

\usepackage{url} %this package should fix any errors with URLs in refs.
\usepackage{lineno}
\usepackage[inline]{trackchanges} %for better track changes. finalnew option will compile document with changes incorporated.
\usepackage{soul}
\linenumbers
\makeatletter
\@ifpackageloaded{caption}{}{\usepackage{caption}}
\AtBeginDocument{%
\ifdefined\contentsname
  \renewcommand*\contentsname{Table of contents}
\else
  \newcommand\contentsname{Table of contents}
\fi
\ifdefined\listfigurename
  \renewcommand*\listfigurename{List of Figures}
\else
  \newcommand\listfigurename{List of Figures}
\fi
\ifdefined\listtablename
  \renewcommand*\listtablename{List of Tables}
\else
  \newcommand\listtablename{List of Tables}
\fi
\ifdefined\figurename
  \renewcommand*\figurename{Figure}
\else
  \newcommand\figurename{Figure}
\fi
\ifdefined\tablename
  \renewcommand*\tablename{Table}
\else
  \newcommand\tablename{Table}
\fi
}
\@ifpackageloaded{float}{}{\usepackage{float}}
\floatstyle{ruled}
\@ifundefined{c@chapter}{\newfloat{codelisting}{h}{lop}}{\newfloat{codelisting}{h}{lop}[chapter]}
\floatname{codelisting}{Listing}
\newcommand*\listoflistings{\listof{codelisting}{List of Listings}}
\makeatother
\makeatletter
\makeatother
\makeatletter
\@ifpackageloaded{caption}{}{\usepackage{caption}}
\@ifpackageloaded{subcaption}{}{\usepackage{subcaption}}
\makeatother

\ifLuaTeX
  \usepackage{selnolig}  % disable illegal ligatures
\fi
\usepackage{bookmark}

\IfFileExists{xurl.sty}{\usepackage{xurl}}{} % add URL line breaks if available
\urlstyle{same} % disable monospaced font for URLs
\hypersetup{
  pdftitle={Le rôle des divinités dans les Métamorphoses d'Ovide},
  pdfauthor={Axelle Penture},
  pdfkeywords={Ovide, Les
Métamorphoses, Divinités, Minerve, Apollon, Jupiter},
  colorlinks=true,
  linkcolor={blue},
  filecolor={Maroon},
  citecolor={Blue},
  urlcolor={Blue},
  pdfcreator={LaTeX via pandoc}}


\journalname{None}

\draftfalse

\begin{document}
\title{Le rôle des divinités dans les \textbf{Métamorphoses} d'Ovide}

\authors{Axelle Penture\affil{1}}
\affiliation{1}{Paris Nanterre Université, }
\correspondingauthor{Axelle Penture}{pentureaxelle@gmail.com}


\begin{abstract}
none
\end{abstract}

\section*{Plain Language Summary}
none




\#\#Proposition de plan

\#\#\#Plan du M1

I. La structure des Métamorphoses : le travail de Gilles Tronchet 1. Le
découpage des Métamorphoses Considérations générales La définition de
l'épisode Le premier découpage de Gilles Tronchet 2. Les typologies des
métamorphoses Métamorphose : définitions Typologie des métamorphoses par
catégorie et par genre 3. Les schémas narratifs actanciels ou «
algorithmes » II. Préparation à l'exploration numérique de la structure
dans les Métamorphoses 1. Extraction de la méthode et des données Les
restrictions de la méthode de Gilles Tronchet dans le cadre d'une
démarche numérique Vers une exploration numérique 2. Exploration
numérique : adapter les données Travail autour des classifications
Traitement numérique du texte 3. Balayage lexical préliminaire avec
Voyant Tools Exploration de l'oeuvre Exploration des épisodes des
métamorphoses III. Exploration numérique des Métamorphoses : réseaux et
échelles de la métamorphose. 1. Analyse structurelle globale
Visualisation et analyse de la place des métamorphoses au sein du texte
selon la caractérisation de Tronchet Analyse quantitative des vers
dédiés à la métamorphose à l'échelle de l'oeuvre 2. Analyse de réseaux
Qui transforme qui ? En quoi transforme qui ? 3. Modèle d'analyse d'une
métamorphose

\#\#\#Nouvelle proposition (pour l'instant en point clés)

\#\#\#Introduction - état de l'art sur les études des divinités dans les
récits d'Ovide (quantitativement et thématiquement); état des travaux
sur les Métamorphoses, parcellaires et problème de la taille du corpus.
Prosition d'une approche numérique pour l'étude transversale à travers
un fil rouge des divinités et de leur dynamique dans les récits de
mtamorphoses, essai d'une méthode numérique à vocation d'analyse
littéraire. - Mention des précédents travaux sur lesquels je me base
pour les analyses : Guillaume Tronchet, les découpages des éditions que
j'utilise, mes travaux de l'année passée et le fichier XML de Nadine
Rakofsky.

\#\#\#Fils rouges (reprendre les notes de Freeform)

\begin{center}\rule{0.5\linewidth}{0.5pt}\end{center}

\section{Introduction (docuement
type)}\label{introduction-docuement-type}

\textsubscript{Source:
\href{https://pax3l.github.io/m2_ovid_deities_analysis_for_quarto/index.qmd.html}{Article
Notebook}}

\phantomsection\label{cell-fig-timeline}
\begin{figure}[H]

\centering{

\includegraphics{index_files/figure-pdf/fig-timeline-1.pdf}

}

\caption{\label{fig-timeline}Timeline of recent earthquakes on La Palma}

\end{figure}%

\textsubscript{Source:
\href{https://pax3l.github.io/m2_ovid_deities_analysis_for_quarto/index.qmd.html}{Article
Notebook}}

\textsubscript{Source:
\href{https://pax3l.github.io/m2_ovid_deities_analysis_for_quarto/index.qmd.html}{Article
Notebook}}

\begin{equation}\phantomsection\label{eq-poisson}{
p(x) = \frac{e^{-\lambda} \lambda^{x}}{x !}
}\end{equation}

where \(\lambda\) is the rate of eruptions per year. Using
Equation~\ref{eq-poisson}, the probability of an eruption in the next
\(t\) years can be calculated.

\begin{longtable}[]{@{}ll@{}}
\caption{Recent historic eruptions on La
Palma}\label{tbl-history}\tabularnewline
\toprule\noalign{}
Name & Year \\
\midrule\noalign{}
\endfirsthead
\toprule\noalign{}
Name & Year \\
\midrule\noalign{}
\endhead
\bottomrule\noalign{}
\endlastfoot
Current & 2021 \\
Teneguía & 1971 \\
Nambroque & 1949 \\
El Charco & 1712 \\
Volcán San Antonio & 1677 \\
Volcán San Martin & 1646 \\
Tajuya near El Paso & 1585 \\
Montaña Quemada & 1492 \\
\end{longtable}

Table~\ref{tbl-history} summarises the eruptions recorded since the
colonization of the islands by Europeans in the late 1400s.

\begin{figure}

\centering{

\includegraphics{images/la-palma-map.png}

}

\caption{\label{fig-map}Map of La Palma}

\end{figure}%

La Palma is one of the west most islands in the Volcanic Archipelago of
the Canary Islands (Figure~\ref{fig-map}).

\begin{figure}[H]

\centering{

\includegraphics{index_files/figure-latex/notebooks-explore-earthquakes-fig-spatial-plot-output-1.png}

}

\caption{\label{fig-spatial-plot}Locations of earthquakes on La Palma
since 2017}

\end{figure}%

\textsubscript{Source:
\href{https://pax3l.github.io/m2_ovid_deities_analysis_for_quarto/notebooks/explore-earthquakes-preview.html\#cell-fig-spatial-plot}{Explore
Earthquakes}}

Figure~\ref{fig-spatial-plot} shows the location of recent Earthquakes
on La Palma.

\section{Data \& Methods}\label{sec-data-methods}

\section{Conclusion}\label{conclusion}

\section*{References}\label{references}
\addcontentsline{toc}{section}{References}

\phantomsection\label{refs}
\begin{CSLReferences}{0}{1}
\vspace{1em}

\end{CSLReferences}




\end{document}
