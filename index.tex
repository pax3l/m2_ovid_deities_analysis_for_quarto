% Options for packages loaded elsewhere
\PassOptionsToPackage{unicode}{hyperref}
\PassOptionsToPackage{hyphens}{url}
\PassOptionsToPackage{dvipsnames,svgnames,x11names}{xcolor}
%
\documentclass[
  letterpaper,
  DIV=11,
  numbers=noendperiod]{scrreprt}

\usepackage{amsmath,amssymb}
\usepackage{iftex}
\ifPDFTeX
  \usepackage[T1]{fontenc}
  \usepackage[utf8]{inputenc}
  \usepackage{textcomp} % provide euro and other symbols
\else % if luatex or xetex
  \usepackage{unicode-math}
  \defaultfontfeatures{Scale=MatchLowercase}
  \defaultfontfeatures[\rmfamily]{Ligatures=TeX,Scale=1}
\fi
\usepackage{lmodern}
\ifPDFTeX\else  
    % xetex/luatex font selection
\fi
% Use upquote if available, for straight quotes in verbatim environments
\IfFileExists{upquote.sty}{\usepackage{upquote}}{}
\IfFileExists{microtype.sty}{% use microtype if available
  \usepackage[]{microtype}
  \UseMicrotypeSet[protrusion]{basicmath} % disable protrusion for tt fonts
}{}
\makeatletter
\@ifundefined{KOMAClassName}{% if non-KOMA class
  \IfFileExists{parskip.sty}{%
    \usepackage{parskip}
  }{% else
    \setlength{\parindent}{0pt}
    \setlength{\parskip}{6pt plus 2pt minus 1pt}}
}{% if KOMA class
  \KOMAoptions{parskip=half}}
\makeatother
\usepackage{xcolor}
\setlength{\emergencystretch}{3em} % prevent overfull lines
\setcounter{secnumdepth}{5}
% Make \paragraph and \subparagraph free-standing
\makeatletter
\ifx\paragraph\undefined\else
  \let\oldparagraph\paragraph
  \renewcommand{\paragraph}{
    \@ifstar
      \xxxParagraphStar
      \xxxParagraphNoStar
  }
  \newcommand{\xxxParagraphStar}[1]{\oldparagraph*{#1}\mbox{}}
  \newcommand{\xxxParagraphNoStar}[1]{\oldparagraph{#1}\mbox{}}
\fi
\ifx\subparagraph\undefined\else
  \let\oldsubparagraph\subparagraph
  \renewcommand{\subparagraph}{
    \@ifstar
      \xxxSubParagraphStar
      \xxxSubParagraphNoStar
  }
  \newcommand{\xxxSubParagraphStar}[1]{\oldsubparagraph*{#1}\mbox{}}
  \newcommand{\xxxSubParagraphNoStar}[1]{\oldsubparagraph{#1}\mbox{}}
\fi
\makeatother


\providecommand{\tightlist}{%
  \setlength{\itemsep}{0pt}\setlength{\parskip}{0pt}}\usepackage{longtable,booktabs,array}
\usepackage{calc} % for calculating minipage widths
% Correct order of tables after \paragraph or \subparagraph
\usepackage{etoolbox}
\makeatletter
\patchcmd\longtable{\par}{\if@noskipsec\mbox{}\fi\par}{}{}
\makeatother
% Allow footnotes in longtable head/foot
\IfFileExists{footnotehyper.sty}{\usepackage{footnotehyper}}{\usepackage{footnote}}
\makesavenoteenv{longtable}
\usepackage{graphicx}
\makeatletter
\newsavebox\pandoc@box
\newcommand*\pandocbounded[1]{% scales image to fit in text height/width
  \sbox\pandoc@box{#1}%
  \Gscale@div\@tempa{\textheight}{\dimexpr\ht\pandoc@box+\dp\pandoc@box\relax}%
  \Gscale@div\@tempb{\linewidth}{\wd\pandoc@box}%
  \ifdim\@tempb\p@<\@tempa\p@\let\@tempa\@tempb\fi% select the smaller of both
  \ifdim\@tempa\p@<\p@\scalebox{\@tempa}{\usebox\pandoc@box}%
  \else\usebox{\pandoc@box}%
  \fi%
}
% Set default figure placement to htbp
\def\fps@figure{htbp}
\makeatother
% definitions for citeproc citations
\NewDocumentCommand\citeproctext{}{}
\NewDocumentCommand\citeproc{mm}{%
  \begingroup\def\citeproctext{#2}\cite{#1}\endgroup}
\makeatletter
 % allow citations to break across lines
 \let\@cite@ofmt\@firstofone
 % avoid brackets around text for \cite:
 \def\@biblabel#1{}
 \def\@cite#1#2{{#1\if@tempswa , #2\fi}}
\makeatother
\newlength{\cslhangindent}
\setlength{\cslhangindent}{1.5em}
\newlength{\csllabelwidth}
\setlength{\csllabelwidth}{3em}
\newenvironment{CSLReferences}[2] % #1 hanging-indent, #2 entry-spacing
 {\begin{list}{}{%
  \setlength{\itemindent}{0pt}
  \setlength{\leftmargin}{0pt}
  \setlength{\parsep}{0pt}
  % turn on hanging indent if param 1 is 1
  \ifodd #1
   \setlength{\leftmargin}{\cslhangindent}
   \setlength{\itemindent}{-1\cslhangindent}
  \fi
  % set entry spacing
  \setlength{\itemsep}{#2\baselineskip}}}
 {\end{list}}
\usepackage{calc}
\newcommand{\CSLBlock}[1]{\hfill\break\parbox[t]{\linewidth}{\strut\ignorespaces#1\strut}}
\newcommand{\CSLLeftMargin}[1]{\parbox[t]{\csllabelwidth}{\strut#1\strut}}
\newcommand{\CSLRightInline}[1]{\parbox[t]{\linewidth - \csllabelwidth}{\strut#1\strut}}
\newcommand{\CSLIndent}[1]{\hspace{\cslhangindent}#1}

\usepackage{ragged2e}
\setlength{\emergencystretch}{3em}
\raggedbottom
\raggedright
\KOMAoption{captions}{tableheading}
\makeatletter
\@ifpackageloaded{bookmark}{}{\usepackage{bookmark}}
\makeatother
\makeatletter
\@ifpackageloaded{caption}{}{\usepackage{caption}}
\AtBeginDocument{%
\ifdefined\contentsname
  \renewcommand*\contentsname{Table of contents}
\else
  \newcommand\contentsname{Table of contents}
\fi
\ifdefined\listfigurename
  \renewcommand*\listfigurename{List of Figures}
\else
  \newcommand\listfigurename{List of Figures}
\fi
\ifdefined\listtablename
  \renewcommand*\listtablename{List of Tables}
\else
  \newcommand\listtablename{List of Tables}
\fi
\ifdefined\figurename
  \renewcommand*\figurename{Figure}
\else
  \newcommand\figurename{Figure}
\fi
\ifdefined\tablename
  \renewcommand*\tablename{Table}
\else
  \newcommand\tablename{Table}
\fi
}
\@ifpackageloaded{float}{}{\usepackage{float}}
\floatstyle{ruled}
\@ifundefined{c@chapter}{\newfloat{codelisting}{h}{lop}}{\newfloat{codelisting}{h}{lop}[chapter]}
\floatname{codelisting}{Listing}
\newcommand*\listoflistings{\listof{codelisting}{List of Listings}}
\makeatother
\makeatletter
\makeatother
\makeatletter
\@ifpackageloaded{caption}{}{\usepackage{caption}}
\@ifpackageloaded{subcaption}{}{\usepackage{subcaption}}
\makeatother

\usepackage{bookmark}

\IfFileExists{xurl.sty}{\usepackage{xurl}}{} % add URL line breaks if available
\urlstyle{same} % disable monospaced font for URLs
\hypersetup{
  pdftitle={Le rôle des divinités dans les Métamorphoses d'Ovide : étude des cas de Minerve, Apollon et Jupiter},
  pdfauthor={Axelle Penture},
  pdfkeywords={Ovide, Les
Métamorphoses, Divinités, Minerve, Apollon, Jupiter},
  colorlinks=true,
  linkcolor={blue},
  filecolor={Maroon},
  citecolor={Blue},
  urlcolor={Blue},
  pdfcreator={LaTeX via pandoc}}


\title{Le rôle des divinités dans les \emph{Métamorphoses} d'Ovide :
étude des cas de Minerve, Apollon et Jupiter}
\author{Axelle Penture}
\date{2025-03-07}

\begin{document}
\maketitle
\begin{abstract}
none
\end{abstract}

\renewcommand*\contentsname{Table of contents}
{
\hypersetup{linkcolor=}
\setcounter{tocdepth}{2}
\tableofcontents
}

\bookmarksetup{startatroot}

\chapter{\texorpdfstring{Le rôle des divinités dans les
\textbf{Métamorphoses}
d'Ovide}{Le rôle des divinités dans les Métamorphoses d'Ovide}}\label{le-ruxf4le-des-divinituxe9s-dans-les-muxe9tamorphoses-dovide}

none

\hfill\break

\section{\texorpdfstring{\textbf{(I.) Fondations : corpus, méthodes et
données}}{(I.) Fondations : corpus, méthodes et données}}\label{i.-fondations-corpus-muxe9thodes-et-donnuxe9es}

\begin{itemize}
\tightlist
\item
  \textbf{1. Constitution du corpus numérique}

  \begin{itemize}
  \tightlist
  \item
    Numérisation et encodage des \emph{Métamorphoses} : travail de
    Nadine Rakofsky et dépassement
  \item
    Identification et extraction des mentions divines : pré travail
    d'encodage
  \item
    Critères et limites méthodologiques
  \end{itemize}
\item
  \textbf{2. Méthodologie d'analyse }

  \begin{itemize}
  \tightlist
  \item
    Théorie des graphes appliquée aux textes \& construction des réseaux
    de personnages
  \item
    Mesures de centralité et d'influence
  \item
    Présentation des résultats avec un outil de navigation dans le texte
    : édition EVT
  \end{itemize}
\item
  \textbf{3. Minerve, Apollon, Jupiter : portraits établis}

  \begin{itemize}
  \tightlist
  \item
    Origine et figure de Minerve chez Ovide
  \item
    Origine et figure d'Apollon chez Ovide
  \item
    Origine et figure de Jupiter chez Ovide
  \end{itemize}
\end{itemize}

\section{\texorpdfstring{\textbf{(II.) Exploration des données : schémas
et
structures}}{(II.) Exploration des données : schémas et structures}}\label{ii.-exploration-des-donnuxe9es-schuxe9mas-et-structures}

\begin{itemize}
\tightlist
\item
  \textbf{1. Analyse structurelle globale : profils quantitatifs des
  divinités}

  \begin{itemize}
  \tightlist
  \item
    Statistiques descriptives des divinités choisies dans l'architecture
    de l'œuvre
  \item
    Corrélations entre livres, épisodes et interventions divines
  \item
    Visualisation des clusters et typologie des épisodes (graphiques de
    distribution et heatmaps)
  \end{itemize}
\item
  \textbf{2. Réseaux relationnels et interactions}

  \begin{itemize}
  \tightlist
  \item
    Méthodologie d'analyse de réseau avec R
  \item
    Cartographie des interactions divines/Graphes de co-occurrence et
    d'interactions
  \item
    Centralité et influence narrative
  \end{itemize}
\end{itemize}

\section{\texorpdfstring{\textbf{(III.)Interprétation \&
analyse}}{(III.)Interprétation \& analyse}}\label{iii.interpruxe9tation-analyse}

\begin{itemize}
\tightlist
\item
  Relecture des portraits divins :

  \begin{itemize}
  \tightlist
  \item
    Jupiter (- Fréquence et typologie des métamorphoses jupitériennes

    \begin{itemize}
    \tightlist
    \item
      Différences entre pouvoir théorique et pratiques narrative)
    \end{itemize}
  \item
    Apollon (//)
  \item
    Minerve (//)
  \end{itemize}
\item
  Études d'épisodes : entre données et interprétations

  \begin{itemize}
  \tightlist
  \item
    \emph{Arachné} : Minerve
  \item
    \emph{Daphné} : Apollon
  \item
    \emph{Europe} : Jupiter (à fixer avec certitude selon les articles
    de la biblio)
  \end{itemize}
\item
  Synthèse

  \begin{itemize}
  \tightlist
  \item
    Ce que les données révèlent des divinités
  \item
    Apports, limites et perspectives de l'approche numérique
  \end{itemize}
\end{itemize}

\section{Conclusion}\label{conclusion}

\begin{itemize}
\tightlist
\item
  Synthèse des résultats quantitatifs et qualitatifs
\item
  Confirmation des profils des divinités et précisions/apports de
  l'approche numérique
\item
  Apports méthodologiques : reproductibilité et nouvelles perspectives
\end{itemize}

\section{\texorpdfstring{\emph{Annexes}}{Annexes}}\label{annexes}

\begin{itemize}
\item
  Annexe 1 : description de l'entrepôt GitHub (//Readme)
\item
  Annexe 2 : Corpus des mentions des divinités par livre
\item
  Annexe 3 : Code R principal et visualisations
\item
  Annexe 4 : Visualisation.s de réseau détaillée.s
\item
  Bibliographie
\end{itemize}

\bookmarksetup{startatroot}

\chapter{Introduction}\label{introduction}

\begin{itemize}
\tightlist
\item
  État de l'art sur les études des divinités dans les textes d'Ovide
  (quantitativement et thématiquement); état des travaux sur les
  \emph{Métamorphoses}, parcellaires et problème de la taille du corpus.
  Proposition d'une approche numérique pour l'étude transversale à
  travers un fil rouge des divinités et de leur dynamique dans les
  récits de mtamorphoses, essai d'une méthode numérique à vocation
  d'analyse littéraire.
\item
  Mention des précédents travaux sur lesquels je me base pour les
  analyses : Guillaume Tronchet, les découpages des éditions que
  j'utilise, mes travaux de l'année passée et le fichier XML de Nadine
  Rakofsky.
\end{itemize}

\phantomsection\label{refs}
\begin{CSLReferences}{1}{0}
\bibitem[\citeproctext]{ref-bakhouche_larchitecture_2019}
Bakhouche, Béatrice. {``L'architecture des \emph{Métamorphoses}
d'Ovide.''} \emph{Giornale Italiano di Filologia} 71 (January 2019):
235--80. \url{https://doi.org/10.1484/J.GIF.5.118468}.

\bibitem[\citeproctext]{ref-consortium_tei_nodate}
Consortium, The TEI. {``The {TEI} Guidelines,''} n.d.

\bibitem[\citeproctext]{ref-driscoll_digital_2016}
Driscoll, Matthew James, and Elena Pierazzo, eds. \emph{Digital
Scholarly Editing: Theories and Practices}. 1st ed. Vol. 4. Digital
Humanities Series. Cambridge, {UK}: Open Book Publishers, 2016.
\url{https://doi.org/10.11647/OBP.0095}.

\bibitem[\citeproctext]{ref-jockers_macroana_2_2013}
Jockers, Matthew L. \emph{Macroanalysis: Digital Methods and Literary
History}. Urbana: University of Illinois Press, 2013.

\bibitem[\citeproctext]{ref-mathieu-colas_dictionnaire_nodate}
Mathieu-Colas, Michel. {``Dictionnaire des noms de divinités,''} n.d.

\bibitem[\citeproctext]{ref-moretti_graphs_2007}
Moretti, Franco. \emph{Graphs, Maps, Trees: Abstract Models for Literary
History}. Paperback edition. London New York: Verso, 2007.

\bibitem[\citeproctext]{ref-piper_enumerations_2018}
Piper, Andrew. \emph{Enumerations: Data and Literary Study}. University
of Chicago Press, 2018.
\url{https://doi.org/10.7208/chicago/9780226568898.001.0001}.

\bibitem[\citeproctext]{ref-segal_jupiter_2025}
Segal, Charles. {``Jupiter in Ovid's "Metamorphoses",''} 2025.

\bibitem[\citeproctext]{ref-segel_narrative_2010}
Segel, E, and J Heer. {``Narrative Visualization: Telling Stories with
Data.''} \emph{{IEEE} Transactions on Visualization and Computer
Graphics} 16, no. 6 (November 2010): 1139--48.
\url{https://doi.org/10.1109/TVCG.2010.179}.

\bibitem[\citeproctext]{ref-tronchet_metamorphose_1997}
Tronchet, Gilles. {``La Métamorphose à l'œuvre : Recherches Sur La
Poétique d'ovide Dans "Les Métamorphoses".''} These de doctorat, Reims,
1997. \url{https://www.theses.fr/1997REIML002}.

\end{CSLReferences}

\bookmarksetup{startatroot}

\chapter{\textbar{} Partie 1 : Fondations : corpus, méthodes et
données}\label{partie-1-fondations-corpus-muxe9thodes-et-donnuxe9es}

\section{Constitution du corpus}\label{constitution-du-corpus}

\subsection{Corpus matériel et numérique : travail de Nadine Rakofsky et
dépassement}\label{corpus-matuxe9riel-et-numuxe9rique-travail-de-nadine-rakofsky-et-duxe9passement}

Le travail numérique que je réalise requiert une vérification du travail
avec un corpus manuscrit. À la question du choix des éditions des
\emph{Métamorphoses}, je réponds donc avec les trois ouvrages de la
collection des Belles Lettres publiés entre 1965 et 1966, dont le texte
fut établi par Georges Lafaye. Ils ont été ma référence de vérification
pour l'encodage et mon support d'analyse littéraire pour le présent
travail.

En ce qui concerne le corpus pour une exploration numérique, il fallait,
dans l'idéal, partir d'un fichier un peu plus élaboré qu'un fichier en
plain text, avec un premier niveau d'encodage qui au moins séparait les
livres et les épisodes au sein des livres. C'est le genre de fichier
proposé à l'exportation par la Perseus Digital Library. Cependant, grâce
à l'intermédiaire d'Aurélien Berra, j'ai pu avoir entre les mains le
fichier de travail d'une ancienne élève du master, Nadine Rakofsky, qui
a travaillé sur \href{https://classnum.hypotheses.org/6414}{le genre
d'Iphis au livre IX des Métamorphoses}, ainsi que sur
\href{https://classnum.hypotheses.org/6455}{la question du secret sur
l'ensemble des quinze livres}. Son fichier prend pour point de départ
celui de la \href{https://www.perseus.tufts.edu/hopper/}{Perseus Digital
Library}, qui propose à l'exportation tous ces textes en format XML. Le
texte choisi pour l'encodage n'est pas celui établi par Georges Lafaye,
mais celui d'Hugo Magnus, qui publie le texte établi en 1892 en
Allemagne, aux éditions Friedrich Andreas Perthes.

L'édition présente un encodage de la structure simple, à savoir le
découpage des livres et la séparation des épisodes narratifs au sein des
livres. Nadine Rakofsky, dans le cadre de ses recherches, y a ajouté le
balisage des personnages en tant que narrateur ou personnage de l'action
tout au long des quinze livres. Sa base de travail s'est avérée parfaite
pour le présent travail, car son encodage permet une analyse de la prise
de parole des divinités choisies. De plus, venir enrichir un fichier
s'inscrit dans l'esprit de ma recherche qui, bien qu'elle s'intéresse
uniquement à trois divinités ici, a pour perspective d'ouvrir la voie à
une analyse d'un ensemble plus exhaustif des divinités dans les
Métamorphoses. Ce texte fait lui-même l'objet d'enrichissements
antérieurs à celui de Nadine Rakofsky, en vue d'une amélioration de
l'édition numérique proposée sur
\href{https://scaife.perseus.org/}{Scaife Viewer}.

\subsection{Identification et extraction des mentions divines : pré
travail
d'encodage}\label{identification-et-extraction-des-mentions-divines-pruxe9-travail-dencodage}

Avant d'entrer dans le fichier et de modifier le cœur du texte, il me
faut déterminer ce que je vais encoder, et comment. Le choix des
divinités a été présenté en introduction : je m'attache ici à expliquer
quelles mentions de divinités sont encodées et sous quelle forme dois-je
pouvoir les appeler dans le fichier XML avec mes scripts R.

Les mentions des divinités, dans un sens large, se déploient, comme
celles des personnages, entre la mention directe et explicite de leur
prénom, à celle plus subtile d'un pronom de rappel ou adjectif, dont le
sens dépend du contexte dans lequel il est employé. Entre ces deux
extrémités se situent diverses périphrases, fréquemment liées à la
généalogie (fils de Saturne), ou bien mentionnant simplement la nature
du personnage évoqué précédemment (le dieu). Cependant, les
\emph{Métamorphoses} sont une œuvre longue, et très riche en
personnages, avec environ 270 recensés dans l'index nominum de l'édition
Budé. La longueur globale du récit et les enchâssements des épisodes
appellent à une clarification constante pour être compris du lecteur, ce
qui passe nécessairement par l'emploi régulier de qualificatifs clairs
quant aux personnages évoqués. Partant de ce constat et dans le souci
pratique du cadre d'un travail de mémoire, j'ai donc fait le choix
d'encoder uniquement les mentions de noms propres des divinités et les
périphrases les désignants, récurrentes ou non. Les occurrences des noms
propres, peu importe leur cas, sont déjà répertoriées par l'édition
physique des \emph{Métamorphoses}, dans l'\emph{Index nominum}. Les
mentions des périphrases ont été relevées par la lecture suivie du
texte, et ont fait l'objet d'une vérification par recherche de mots-clés
dans le texte après un premier encodage.

Pour ce qui est du balisage en lui-même, j'ai d'abord cherché à créer
une nouvelle balise qui marquerait de façon explicite la recherche des
divinités, sobrement intitulé . Cependant, le processus de création
d'une balise est relativement long pour des résultats atteignables avec
celles déjà existantes. Je me suis donc finalement tournée vers les
possibilités offertes par la balise
\texttt{\textless{}persname\textgreater{}}, qui propose un ensemble
d'attributs tout à fait approprié pour mes identifications. Comme elle
n'avait pas été utilisée par Nadine Rakofsky, son emploi simplifiait
également mon script R pour la future analyse des données.

J'associe à cette balise trois attributs. Le premier attribut,
\texttt{ref}, permet d'identifier la divinité dont il est question dans
le passage. Les divinités sont identifiées par les trois premières
lettres de leur nom latin : « MIN » pour Minerve, « APO » pour Apollon
et « IUP » pour Jupiter. Il est aisé de poursuivre cette liste pour
n'importe quelle autre divinité. Ensuite, l'attribut \texttt{type} qui,
selon la définition des
\href{https://www.tei-c.org/release/doc/tei-p5-doc/fr/html/ref-persName.html}{TEI
Guidelines}, « caractérise l'élément en utilisant n'importe quel système
ou typologie de classification approprié ». Ici, il y a deux catégories
possibles pour cet attribut : soit le passage dans lequel la mention de
la divinité apparaît correspond à un passage direct de métamorphose, tel
qu'ils ont été établi dans
\href{https://docs.google.com/spreadsheets/d/1n_TWKsEPywmv3aVzptd-_cRNn-tIxXZQ3tHtt0w_TJE/edit?gid=2010180086\#gid=2010180086}{le
découpage réalisé l'an dernier}. Les passages correspondants portent
l'étiquette `metamorphosis'. Tous les autres passages sont par défaut
des passages narratifs, donc étiquetés `narrative'. Enfin, l'attribut
\texttt{ana} permet de faire la distinction du rôle tenu par la divinité
dans l'action qui se joue autour. L'année précédente, le travail de
thèse de Gilles Tronchet\footnote{Tronchet, {``La Métamorphose à
  l'œuvre.''}}qui distinguait les épisodes et les métamorphoses,
retravaillé par moi-même avec la nomination et la délimitation des vers
des acteurs, a permis de mettre en évidence les rôles des personnages
impliqués des métamorphoses, distinguant les acteurs et les objets.
Aussi, cet attribut peut accueillir quatre informations :

\begin{itemize}
\tightlist
\item
  `act', qui désigne le rôle d'acteur de la métamorphose qui se joue;
\item
  `obj', qui désigne le rôle d'objet de la métamorphose qui se joue;
\item
  `auto', qui désigne le cas où la divinité est à la actrice et objet
  d'une métamorphose, c'est-à-dire qu'elle se transforme elle-même, ce
  qui est cas courant chez les dieux;
\item
  `NA', désigne les cas où le nom de la divinité est employé à des fins
  qui ne correspondent à aucun des cas cités précédemment.
\end{itemize}

Avec ces trois attributs, j'ai balisé 337 occurrences avec la balise
\texttt{\textless{}persname\textgreater{}} sur l'ensemble des quinze
livres des Métamorphoses.

\subsection{Critères et limites
méthodologiques}\label{crituxe8res-et-limites-muxe9thodologiques}

Le choix des critères a reposé sur mes questions directrices, mais aussi
sur mon travail précédent. Dans la mesure où la délimitation des
épisodes, des espaces de métamorphoses, des acteurs et des objets, pour
ne citer que ces éléments, ont été réalisé l'année précédente, j'avais à
ma disposition beaucoup d'éléments pour penser à un encodage plus direct
dans le texte en vue d'une visualisation globale. Il s'agissait ici de
croiser ces éléments déjà connus avec ceux, plus propres à la question
du travail de mémoire, des mentions de divinités.

Le choix restreint des mentions de noms propres et périphrases, mais
surtout de l'exclusion des pronoms de rappel et des pronoms adjectifs
est principalement dû aux contraintes de temps. La recherche de
cohérence et de clarté évoquée précédemment, le travail préalable sur
les épisodes ainsi que l'existence préalable d'index des noms propres
réduisaient la quantité d'oubli majeur potentiel. De plus, une partie de
ma réflexion cherche à proposer une méthode de travail sur le texte, qui
a pour vocation de s'étendre à d'autres divinités et à préciser les
filtres de lecture à appliquer autant aux éléments choisis qu'à
l'ensemble du texte.

Avant même de commencer à encoder le texte, grâce à un coup d'œil aux
données et de par les critères choisis, je m'attends à rencontrer des
mentions d'acteurs hors des épisodes à proprement parler de
métamorphoses. Le problème est résolu dès qu'il est possible d'établir
un lien logique clair entre la mention de la divinité, qu'elle soit
antérieure ou postérieure, et l'épisode. Le lien logique est manifeste,
que ce soit par l'emploi d'un pronom de rappel, d'un pronom adjectif, ou
par la logique narrative de l'épisode.\\
Un cas simple pour illustrer cette situation est celui de Jupiter
lorsqu'il transforme les fourmis d'Égine en hommes à la demande d'Éaque,
au livre VII. La métamorphose est spécifiquement décrite dans les vers
639 à 642. Jupiter est mentionné depuis le début du passage narratif,
que l'on considère qu'il commence au vers 453, (selon le fichier
numérique), au vers 518 (selon le découpage de Gilles Tronchet) ou au
vers 614 (selon le découpage des épisodes de Georges Lafaye). Il est
interpellé dans des contextes qui n'appellent pas de performance de sa
part, et ces mentions sont dans ce cas rattachés à une fonction
narrative. Ce n'est que sa mention au vers 627, dans une prière très
précise d'Éaque concernant les fourmis (« \emph{Totidem, pater optime, »
dixit,/ « Tu mihi da ciues et inania moenia supple.} »)\footnote{Ajoute
  la traduction de Marie Cosnay.} que la périphrase \emph{pater optime}
est encodée comme mention d'un acteur de la métamorphose qui se joue par
la suite. À l'inverse, une de ses mentions précédentes, à savoir «
\emph{Iuppiter o!} » au vers 615, est elle encodée avec une analyse
`NA', car son invocation ici n'appelle ni ne provoque aucune
métamorphose.

Ces décisions voient naître des difficultés et surtout créent des
limites dans le processus de travail. La première difficulté réside dans
la polymorphie des personnages divins : Minerve, Apollon et Jupiter
peuvent apparaître sous forme d'allégorie, de divinité active, ou être
nommés par leurs épithètes. L'encodage vise donc à homogénéiser les
occurrences pour chaque entité tout en préservant les niveaux
d'interprétation. Le choix des mentions encodées apporte une limite
d'exhaustivité à un tel balayage du texte. Une seconde difficulté est
celle du remplacement des mentions dans le contexte narratif. Comme
évoqué dans l'exemple ci-dessus, les délimitations des épisodes sont
différentes entre l'édition numérique, l'édition de Georges Lafaye et le
travail de Gilles Tronchet, qui servent toutes des objectifs différents.
Ainsi, replacer la mention dans un contexte narratif large, à l'échelle
de l'épisode ou du sous-épisode est complexe entre éditions différentes.
Le choix du contexte avec l'attribut \texttt{type} s'est avéré le plus
simple, bien qu'il restreigne la remise en contexte de la mention.

\section{Méthodologies d'analyses}\label{muxe9thodologies-danalyses}

\subsection{Analyse de répartition et approches quantitatives en
humanités
numériques}\label{analyse-de-ruxe9partition-et-approches-quantitatives-en-humanituxe9s-numuxe9riques}

L'analyse de répartition constitue une méthode exploratoire et
descriptive, visant à cartographier la présence, la fonction et
l'évolution d'entités (ici, Minerve, Jupiter, Apollon) dans un corpus.
Historiquement développée en linguistique de corpus, elle s'étend
aujourd'hui aux humanités numériques pour suivre non seulement où
apparaissent les entités, mais aussi comment et en quelles proportions.

Les trois dimensions d'investigation sont les suivantes :

\begin{itemize}
\tightlist
\item
  La répartition dans le texte : avec la fréquence et la densité par
  livre ou segment narratif.
\item
  Le contexte narratif, à travers l'attribut TEI \texttt{type}, qui
  replace la mention dans un contexte soit narratif soit de
  métamorphose.
\item
  La valeur interprétative, à travers \texttt{ana}, qui permet de
  préciser le degré d'analyse interprétative (`act', `obj', `auto' ou
  `NA').
\end{itemize}

La méthode s'appuie sur un encodage XML-TEI conforme aux standards
Consortium\footnote{{``The {TEI} Guidelines.''}}, reposant sur le
marquage \textless persName \ldots\textgreater{} avec les trois
attributs clés décrits en amont :

\begin{itemize}
\tightlist
\item
  \texttt{ref}, qui pointe vers une fiche unique dans la personography ;
\item
  \texttt{type}, qui décrit le cadre au sein de l'épisode ;
\item
  \texttt{ana}, qui indique une annotation interprétative.
\end{itemize}

Exemple d'encodage au vers 563 du livre II des \emph{Métamorphoses} :

\begin{verbatim}
<persName ref="MIN" type="narrative" ana="NA">Minervae</persName>
\end{verbatim}

Ce format permet de générer une base de données pour chaque occurrence,
structurée selon :

\begin{itemize}
\item
  l'unité narrative (livre, épisode) ;
\item
  le type de moment dans l'unité narrative ;
\item
  le type d'intervention, soit la valeur `ana'.
\end{itemize}

Cette systématisation ouvre la voie à des analyses croisées fines (par
exemple le rôle vs segment narratif), complexe si ce n'est impossible
sans encodage standardisé.

\section{Visualisation des répartitions\,: outils, formats et
éclairages}\label{visualisation-des-ruxe9partitions-outils-formats-et-uxe9clairages}

Les données extraites des fichiers TEI sont converties en format
tabulaire, puis traitées dans l'environnement R à l'aide des packages
\texttt{tidyverse} et \texttt{ggplot2}. Trois types de représentations
graphiques sont privilégiées, chacune offrant un éclairage spécifique
sur les dynamiques narratives et symboliques en jeu.

D'abord, des graphiques de densité sont réalisés pour chaque livre ou
segment narratif. Ils visent à représenter la présence cumulative des
divinités dans la structure linéaire du récit, afin d'en identifier les
points de concentration, de dispersion ou d'absence. Ces visualisations
permettent de repérer les moments d'intensification ou de retrait dans
la narration divine.

Ensuite, des barres empilées ou diagrammes circulaires sont mobilisés
pour visualiser la répartition des rôles (\texttt{ana}) associés à une
même figure. Cette représentation permet de comparer les fonctions
dominantes ou marginales que chaque divinité assume dans le texte,
révélant des préférences narratives ou des variations selon les
segments.

Enfin, des matrices de co-occurrence croisant les annotations
\texttt{ana} (fonction) et \texttt{type} (type d'intervention ou d'objet
narratif) sont représentées en heatmaps ou treemaps. Ces visualisations
permettent de mettre en évidence les combinaisons sémantiques
récurrentes, comme par exemple une possible conjonction fréquente ou
épisodique entre \texttt{ana="act"} et \texttt{type="metamorphosis"},
qui signalerait des rôles actifs dans la dynamique des métamorphoses du
texte.

L'ensemble de ces visualisations vise à constituer, pour chaque
divinité, une carte narrative articulant position dans le récit,
fonction, et contexte d'intervention. Ce dispositif relève d'une
approche de \emph{data storytelling} narratif, telle que définie par
Segel et Heer (Segel and Heer\footnote{{``Narrative Visualization.''}}),
qui associe rigueur quantitative et lecture interprétative. Il permet
ainsi d'explorer la trajectoire des dieux et déesses dans le poème non
seulement comme présence textuelle, mais aussi comme entité signifiante
à part entière dans l'économie du récit ovidien.

\subsection{Perspectives critiques et
transdisciplinaires}\label{perspectives-critiques-et-transdisciplinaires}

L'analyse de répartition, sans recourir à une modélisation en graphe,
permet de caractériser quantitativement la présence narrative. Elle
s'inscrit pleinement dans les approches macroscopiques des humanités
numériques (Moretti\footnote{\emph{Graphs, Maps, Trees}.},
p.~53)\footnote{``What do literary maps do \ldots{} First, they are a
  good way to prepare a text for analysis. You choose a unit-walks,
  lawsuits, luxury goods, whatever-find its occurrences, place them in
  space \ldots{} or in other words: you reduce the text to a few
  elements, and abstract them from the narrative fl.ow, and construct a
  new, artificial object like the maps that I have been discussing. And
  with a little luck, these maps will be more than the sum of their
  parts: they will possess `emerging' qualities, which were not visible
  at the lower level.'' Moretti{} p.53} (Jockers\footnote{\emph{Macroanalysis}.}).
Elle entre en dialogue avec\,:

\begin{itemize}
\item
  les analyses de trajectoire des personnages (Piper\footnote{\emph{Enumerations}.})
  ;
\item
  l'étude de l'usage des noms propres dans les corpus ;
\item
  l'édition critique augmentée (\footnote{\textbf{Babeu\_rome\_2011?}};
  Driscoll and Pierazzo\footnote{\emph{Digital Scholarly Editing}.}), où
  l'encodage TEI dynamise la critique textuelle.
\end{itemize}

Cette méthode est particulièrement pertinente lorsque l'encodage des
interactions est incomplet : elle valorise les attributs descriptifs
(\texttt{ref}, \texttt{type}, \texttt{ana}), qui sont disponibles même
en absence de liens explicites.

\subsection{Cadre technique et
reproductibilité}\label{cadre-technique-et-reproductibilituxe9}

L'exploitation des données encodées en XML-TEI repose sur un pipeline
structuré, conçu pour extraire, analyser et visualiser les occurrences
de divinités dans les Métamorphoses d'Ovide selon des critères narratifs
et interprétatifs. Ce processus garantit la traçabilité des opérations
et la reproductibilité scientifique de l'étude. La première étape
consiste à parser les fichiers XML à l'aide de la fonction
\texttt{read\_xml()} du package xml2 en environnement R. Combinée aux
outils de la suite tidyverse, cette lecture permet d'extraire de manière
ciblée les balises \texttt{\textless{}persName\textgreater{}} comportant
les attributs structurants que \texttt{ref}, \texttt{type} et
\texttt{ana}. Chaque balise est associée à son unité narrative de
référence, qu'il s'agisse d'un livre, d'un épisode ou d'un segment
temporel grâce à la numérotation interne.

Les données extraites sont alors organisées dans un data frame
structuré, où chaque ligne correspond à une occurrence balisée, et
chaque colonne à une variable analytique : nom de la figure
(\texttt{ref}), segment ou livre d'apparition, rôle narratif (type),
valeur interprétative (\texttt{ana}), et le cas échéant des variables
temporelles ou contextuelles supplémentaires. Cette tabularisation
constitue une base stable pour les traitements statistiques et
graphiques. Les visualisations sont générées en R à l'aide de ggplot2 et
d'extensions adaptées. Plusieurs types de représentations sont mobilisés
pour rendre compte des différentes dimensions de l'analyse :

\begin{itemize}
\item
  La fonction \texttt{geom\_density()} permet de représenter les
  distributions linéaires des divinités dans la trame des livres, en
  mettant en évidence les pics d'apparition ou les zones de silence.
\item
  Les graphes de répartition des rôles
  (\texttt{geom\_bar(stat\ =\ "count",\ position\ =\ "fill")}) offrent
  une vue comparative des fonctions narratives occupées par chaque
  figure dans le corpus.
\item
  Enfin, les relations croisées entre \texttt{ana} et \texttt{type} sont
  visualisées via des matrices de co-occurrence, construites avec
  \texttt{geom\_tile()} (heatmaps) ou \texttt{treemap()} (arborescences
  pondérées), permettant d'identifier des motifs interprétatifs
  dominants ou des asymétries symboliques.
\end{itemize}

Ces visualisations, loin d'être purement descriptives, constituent un
support essentiel pour l'interprétation scientifique. Elles autorisent
des comparaisons inter-divinités (que nous ne mettrons pas en place
ici), la détection de ruptures narratives ou d'évolutions dans les
représentations, et la mise en évidence de configurations singulières
dans l'usage des figures mythologiques.

Ce pipeline analytique est entièrement consigné dans \href{…}{un fichier
R Markdown}, documentant chaque étape du traitement : lecture,
extraction, nettoyage, structuration, visualisation, interprétation.
Cette approche garantit la transparence méthodologique et la
répétabilité des résultats, conformément aux standards des humanités
numériques. Les scritps d'analyse R sont archivés dans
\href{https://github.com/pax3l/m2_ovid_deities_analysis_for_quarto}{le
dépôt de données Git Hub} en vue de toute reproductibilité, amélioration
et observation sous un encodage mis à jour.

\subsection{Navigation dans le texte et ancrage des données : l'édition
électronique
(EVT)}\label{navigation-dans-le-texte-et-ancrage-des-donnuxe9es-luxe9dition-uxe9lectronique-evt}

Afin d'ancrer les données issues de l'analyse computationnelle dans le
texte littéraire lui-même, une édition électronique enrichie a été créée
avec l'outil EVT (Edition Visualization Technology). Il s'agit d'une
interface web qui permet de consulter le texte encodé (au format
TEI-XML), avec un système de navigation par entités et références
croisées.

L'interface donne accès à :

\begin{itemize}
\item
  Une lecture du texte latin et des annotations des divinités,
  permettant de suivre à la fois la version originale et d'avoir à la
  loupe les objets d'étude ;
\item
  Un index dynamique des personnages encodés, qui permet de filtrer le
  texte par entité et d'explorer ses différentes occurrences ;
\item
  Des cartes de réseau interactives, affichables par livre ou épisode,
  qui sont directement reliées au texte encodé, assurant ainsi la
  traçabilité des données ;
\item
  Des filtres permettant de suivre un personnage dans l'ensemble de
  l'œuvre, par exemple pour reconstituer la trajectoire de Minerve ou
  identifier tous les récits où Jupiter intervient en tant qu'acteur de
  métamorphose.
\end{itemize}

Cette édition vise une double finalité : offrir un outil d'analyse au
chercheur (grâce aux liens avec les données quantitatives) et rendre le
résultat intelligible à un lectorat plus large (enseignants, étudiants,
passionnés de littérature antique). Le dialogue entre quantité et
qualité, distant reading et close reading, trouve ici une expression
concrète. Enfin, cette interface est reliée au dépôt GitHub du projet,
qui documente l'ensemble des choix techniques et théoriques. Cette
transparence vise la reproductibilité scientifique du travail et
encourage une approche collaborative de la recherche en humanités
numériques. transition : Ces méthodes permettent de dresser des profils
quantitatifs des divinités analysées et de comprendre les schémas
narratifs dans lesquels elles s'insèrent. L'étude des réseaux et de leur
structure ouvre alors sur une interprétation qualitative et littéraire,
qui sera développée dans les sections suivantes.

\section{Minerve, Apollon, Jupiter : premiers
portraits}\label{minerve-apollon-jupiter-premiers-portraits}

L'analyse moderne des représentations divines en littérature est
parcellaire et prend comme base de travail des corpus inégaux. Elle est
très souvent recoupée avec l'analyse des rites, des représentations
visuelles, souvent statuaires, qui sont au cœur de la pratique des grecs
et des romains dans leur rapport aux divinités. Les caractéristiques et
attributs de ces divinités se transmettent de façon toujours déclinées
mais très similaires depuis la Grèce archaïque jusqu'à la Rome
classique. Dans le contexte de l'Empire romain, les divinités occupent
une place à la fois symbolique, morale et culturelle. Elles incarnent
des principes abstraits, des forces naturelles ou bien des idéaux
civiques, tout en restant des figures mythologiques au cœur du récit
poétique et philosophique. Les auteurs latins oscillent entre un
traitement narratif traditionnel, comme Virgile dans l'\emph{Énéïde} et
Ovide lui-même, et une réflexion philosophique sur la nature divine. Ces
divinités incarnent des valeurs motrices ou disruptives de la société
romaine --- sagesse, justice, vengeance, guerre --- et justifient ainsi
l'ordre du monde, en expliquant des situations politiques ou
interrogeant la condition humaine. D'autres auteurs, tels que Cicéron
dans le \emph{De Natura Deorum}, ont une approche bien plus critique et
rationaliste, et invitent à une lecture plus philosophique de leur
existence. Le paysage littéraire confère donc aux divinités une double
facette qui témoigne d'une transition dans la population, entre
appropriation du patrimoine religieux, hérité de la Grèce antique, et
outil d'expression culturel et intellectuel. Cette phase de transition
invite à examiner leurs diverses représentations et les différences
notables entre les auteurs. Les différentes époques construisent divers
portraits des divinités, dont les auteurs se sont fait porteurs de la
tradition pour leurs successeurs. Nous ne cherchons pas ici à construire
un portrait exhaustif comparé entre les auteurs des figures de
divinités, axe de lecture qui manque quelque peu de littérature. Il
s'agit plutôt ici de dresser un portrait non exhaustif soulignant les
caractéristiques principales des figures divines chez des auteurs
classiques, afin d'avoir un point de départ et de comparaison avec les
éléments apportés par l'analyse numérique.

Bien que chaque auteur maintienne un rapport personnel au passage de la
République à l'Empire, à Auguste et à la politique de Rome est général,
nous partons du principe que les divinités ne sont pas traitées de façon
totalement opposées d'une époque à l'autre, particulièrement celle que
nous avons choisies pour cette étude, et que leur portrait s'inscrit
dans un ensemble de codes, présents également hors littérature, qui
forment une image continue.

Nous nous intéressons donc aux auteurs contemporains d'Ovide, qui
participent à l'établissement de l'image des divinités dans la
littérature latine de l'époque classique. Parmi eux, Virgile,
inspiration presque explicite pour nombreux évènements des
Métamorphoses. Nous observons aussi les textes de ces contemporains
évidents, tels que Tite-Live, Horace, et Sénèque l'Ancien, malgré un
genre littéraire de prédilection différent. Nous n'oublions pas les
contemporains également liés par le style élégiaque, comme Horace et
Tibulle, ce dernier attestant d'une relation directe avec Ovide dans les
textes. L'établissement des portraits est réalisé grâce à des recherches
des patronymes latins les plus courants dans les bases de données de la
Perseus Digital Library, à travers le nouvel environnement de lecture
\href{https://scaife.perseus.org/}{Scaife Viewer}.

\subsection{Minerve, stratège fière, protectrice des arts et des
techniques}\label{minerve-stratuxe8ge-fiuxe8re-protectrice-des-arts-et-des-techniques}

Elle est désignée dans le Dictionnaire des noms des divinités de Michel
Mathieu-Colas (Mathieu-Colas\footnote{{``Dictionnaire des noms de
  divinités.''}}), sous son avatar le plus simple comme la « déesse de
la pensée, des lettres et des arts; protectrice des corporations et des
métiers ». Sa variante grec, Athéna, y ajoute « des sciences ». Ses
occurrences ne sont pas très nombreuses chez les contemporains d'Ovide.
Elle semble régulièrement mentionnée parmi d'autres divinités, en
description de lieux lui rendant hommage, ou invoquée dans un appel à la
protection des arts, aux lettres, à la raison. Dans l'Hymne homérique à
Athéna, c'est son caractère de protectrice de la ville, et ses capacités
guerrières qui sont soulignées, avec un parallèle à Arès. Chez Virgile,
elle est invoqué à travers les caractéristiques précédemment évoquées
dans le passage qui mène à la construction des armes d'Énée par Vulcain,
d'abord simplement pour évoquer les travaux manuels des femmes
(\emph{cui tolerare colo uitam tenuique Minerua/impositum,}
{[}\ldots{]}, 8, 409-410)\footnote{trad. à placer L'\emph{Énéide}, 8,
  409-410}, puis pour insister sur son caractère guerrier à travers le
rappel de son histoire avec la Gorgone (\emph{aegidaque horriferam,
turbatae Palladis arma,/certatim squamis serpentum auroque
polibant/conexosque anguis ipsamque in pectore diuae/Gorgona, desecto
uertentem lumina collo.})\footnote{trad. à placer L'\emph{Énéide}, ?}.
Les autres auteurs élégiaques la mentionnent peu, et de façon peu
significative. Tite-Live la mentionne dans au moins 20 passages
différents du \emph{Ab urbe condita}, souvent à côté d'une mention de
Jupiter. Elle y est présentée comme la déesse des nombres (\emph{eum
clavum, quia rarae per ea tempora litterae erant, notam numeri annorum
fuisse ferunt eoque Minervae templo dicatam legem, quia numerus Minervae
sit.})\footnote{trad. de Scaife \emph{Ab urbe condita} 7.3.6}, mais le
plus souvent associée à la triade capitoline, avec Jupiter et Junon, et
dans le cadre de mentions à des rites, des offrandes ou des sacrifices,
notamment à l'évocation des Quinquatries au livre XXVII. La déesse, dans
un ensemble de mentions qui prennent peu le temps de la description, se
présente comme gardienne des techniques, des sujets de pensée, et
accessoirement renfort de la stratégie guerrière.

\subsection{Apollon, amoureux transit des arts et des
mortels}\label{apollon-amoureux-transit-des-arts-et-des-mortels}

Le même dictionnaire(Mathieu-Colas\footnote{{``Dictionnaire des noms de
  divinités.''}}) nous livre une description d'Apollon comme le « dieu
de la lumière de la beauté et des arts ». Ses variations Phoebus et
Phébus le lient plus explicitement au Soleil. On retrouve directement
des mentions du Soleil, Hélios, dans des récits de toute nature,
cosmogoniques (\emph{Théogonie}) ou encore pastoraux (\emph{Les
Bucoliques}), mais aussi dans des cadres épiques, des contextes de
guerre ou de conflits. Apollon y est alors associé à la Pythie, sa
dévouée prêtresse au temple de Delphes, connue pour ses oracles décisifs
mais souvent sibyllins. Tite-Live, dans un contexte plus historique, le
mentionne surtout ce cadre, ou en tous cas dans un contexte religieux ou
rituel (\emph{itaque inde consules, ne criminationi locus esset, in
prata Flaminia, ubi nunc aedes Apollinis est --- iam Apollinare
appellabant --- , avocavere senatum})\footnote{\emph{Ab urbe condita},
  III, 63.7}. Les auteurs élégiaques le nomment, sans trop de surprise,
surtout dans des appels à la protection, à la consécration des arts et à
l'inspiration, à côté des Muses. Son rôle est principalement celui d'une
divinité protectrice et d'un oracle de référence.

\subsection{Jupiter, amant invétéré, père des dieux et des
hommes}\label{jupiter-amant-invuxe9tuxe9ruxe9-puxe8re-des-dieux-et-des-hommes}

Jupiter, dans sa forme latine la plus simple, est décrit par
Mathieu-Colas (Mathieu-Colas\footnote{{``Dictionnaire des noms de
  divinités.''}}) comme le « dieu du Ciel, de la lumière, de la foudre,
père et souverain des dieux ». Chez Ovide, il prend ses racines dans le
Zeus de l'\emph{Énéide} de Virgile et celui de l'Iliade et l'Odyssée
d'Homère, mêlant un autoritarisme solennel et une puissance capricieuse,
tantôt bienveillante tantôt destructrice. Le contexte littéraire mais
surtout politique d'Ovide lui fait prendre une position plus importante,
puisqu'il fait de Jupiter le pendant divin d'Auguste, ce qui est
explicitement lisible au livre XV (858--860) :

\begin{quote}
\emph{Iuppiter arces} \emph{temperat aetherias et mundi regna triformis}
\emph{terra sub Augusto est; pater est et rector uterque.}\footnote{\emph{Métamorphoses}
  15, 858--860, trad. de Bakhouche, {``L'architecture des
  \emph{Métamorphoses} d'Ovide''} : « Jupiter règne dans le ciel, la
  terre obéit à Auguste: tous deux sont les pères et les souverains de
  leur empire ».}
\end{quote}

Jupiter a fait l'objet d'une analyse très récente dans les œuvres
principales d'Ovide, ce qui n'est pas le cas dans la bibliographie
accessible sur Minerve et Apollon. Dans « Jupiter in Ovid's
Metamorphoses » (Segal\footnote{{``Jupiter in Ovid's "Metamorphoses".''}})
Charles Ségal propose un portrait de Jupiter dans les
\emph{Métamorphoses} à partir de l'héritage antique de Virgile et
Homère, en faisant des parallèle avec des auteurs contemporains d'Ovide.
Jupiter est alors d'abord « le gardien responsable de l'ordre humain et
divin » (p.~79, Segal\footnote{}). Il se caractérise principalement par
la colère et la violence (p.80 Segal\footnote{}), moteurs principaux des
métamorphoses qu'il provoque. Il incarne une ambivalence, entre un père
colérique et tout-puissant, et un mari trompeur, craintif des
représailles de sa femme. Malgré sa toute-puissance et son commandement
sur l'ensemble des dieux et des hommes, sur lesquels il n'hésite pas à
faire connaître son courroux, il ploie devant le regard de Junon et sa
colère face à ses infidélités. Un coup d'œil à l'index nominum suffit à
confirmer qu'il est le dieu le plus mentionné dans l'ensemble des
livres. Son rôle narratif et symbolique chez Ovide est souligné par
Ségal, pour qui « Ovid's ways of treating the Olympian ruler
self-consciously embodies the contrasting thematic and stylistic levels
of the poem. » (p.83 Segal\footnote{})\footnote{Les moyens qu'Ovide
  emploie pour traiter du souverain de l'Olympe incarnent consciemment
  les thèmes contrastés et les niveaux de style du poème. placer la
  citation en notes. \emph{trad. personnelle}}

Cette première partie a permis d'établir les fondations méthodologiques
nécessaires à une analyse numérique des divinités dans les Métamorphoses
d'Ovide. En nous appuyant sur le travail d'encodage XML-TEI
préalablement réalisé par Nadine Rakofsky, nous avons développé un
protocole d'analyse original qui conjugue rigueur philologique et outils
numériques.

L'encodage de 338 occurrences à travers les quinze livres, organisé
autour de trois attributs structurants (ref, type, ana), constitue
désormais une base de données exploitable pour l'analyse quantitative.
Cette systématisation permet de dépasser les approches traditionnelles,
souvent fragmentaires, de l'étude des divinités ovidiennes. Les choix
méthodologiques opérés --- limitation aux noms propres et périphrases,
distinction entre contexte narratif et métamorphique, typologie des
rôles actantiels --- offrent un cadre d'analyse reproductible et
extensible à d'autres figures divines.

Les premiers portraits établis révèlent déjà des spécificités
intéressantes : Minerve apparaît comme une figure stratégique,
protectrice des arts et des techniques ; Apollon se dessine en divinité
des arts et un oracle de référence ; Jupiter s'impose comme la figure
paternelle ambivalente, oscillant entre autorité souveraine et
vulnérabilité conjugale. Ces caractérisations, issues des sources
antiques et de la critique moderne, constituent notre socle
interprétatif.

L'infrastructure technique mise en place --- pipeline d'extraction R,
visualisations dynamiques, édition électronique EVT --- garantit non
seulement la reproductibilité scientifique de notre démarche, mais ouvre
également des perspectives de recherche collaborative. Cette approche
méthodologique représente un apport significatif aux études ovidiennes,
en proposant une alternative aux lectures parcellaires traditionnelles.

Les fondations étant posées, il convient désormais de mettre à l'épreuve
notre dispositif d'analyse. Les données encodées vont révéler leurs
potentialités heuristiques à travers l'exploration systématique des
répartitions, des fréquences et des contextes d'apparition de nos trois
divinités.

La partie suivante procédera d'abord à une analyse structurelle globale,
en établissant les profils quantitatifs de Minerve, Apollon et Jupiter
dans l'architecture des \emph{Métamorphoses}. Cette exploration révélera
les logiques de distribution narrative, les corrélations entre
interventions divines et segments textuels, ainsi que les schémas
récurrents qui structurent le récit ovidien grâce à des visualisations
spécialisées (graphiques de distribution, heatmaps).

L'enjeu de cette exploration des données est double : valider
empiriquement les portraits divins esquissés et identifier les
structures quantitatives qui orienteront l'interprétation littéraire.
Les résultats obtenus alimenteront directement la troisième partie,
consacrée à la relecture critique des portraits divins et à l'analyse
approfondie d'épisodes emblématiques, où données et interprétations
dialogueront pour révéler ce que l'approche numérique apporte à la
compréhension des \emph{Métamorphoses}.

\bookmarksetup{startatroot}

\chapter{Partie 2 : Exploration des données : schémas et
structures}\label{partie-2-exploration-des-donnuxe9es-schuxe9mas-et-structures}

\bookmarksetup{startatroot}

\chapter{Partie 3 : Interprétation \&
analyse}\label{partie-3-interpruxe9tation-analyse}

\bookmarksetup{startatroot}

\chapter{Conclusion}\label{conclusion-1}

\bookmarksetup{startatroot}

\chapter{Conclusion}\label{conclusion-2}

\bookmarksetup{startatroot}

\chapter{Bibliographie}\label{bibliographie}

\section{\texorpdfstring{Éditions des \emph{Métamorphoses}
d'Ovide}{Éditions des Métamorphoses d'Ovide}}\label{uxe9ditions-des-muxe9tamorphoses-dovide}

\section{Oeuvres antiques}\label{oeuvres-antiques}

\section{Outils numériques}\label{outils-numuxe9riques}

\section{Ouvrages et articles
scientifiques}\label{ouvrages-et-articles-scientifiques}




\end{document}
